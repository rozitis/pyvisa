% -*- mode: LaTeX; coding: us-ascii; ispell-local-dictionary: "british"; -*-
\documentclass{howto}

\title{PyVISA\\[0.5ex]\large measurement and test equipment control with Python}
\release{0.9}

\author{Torsten Bronger}
\authoraddress{
	Aachen, Germany\\
	\email{bronger@physik.rwth-aachen.de}
}

\date{17 June 2005}

\makeindex

\begin{document}

\maketitle

\ifhtml
\chapter*{Front Matter\label{front}}
\fi

Copyright \copyright{} 2005 Torsten Bronger.\\
Permission is granted to copy, distribute and/or modify this document under the
terms of the GNU Free Documentation License, Version~1.2 or any later version
published by the Free Software Foundation; with no Invariant Sections, no
Front-Cover Texts, and no Back-Cover Texts.  A copy of the license is included
as a separate file \file{LICENSE} in the PyVISA distribution.

\begin{abstract}

\noindent
This document covers the PyVISA module.  It bases on the VISA library and
implements control of measurement devices in a convenient way.
\end{abstract}

\tableofcontents

\declaremodule{extension}{pyvisa}
  \platform{Linux,Windows}
\modulesynopsis{Controlling measurement and test equipment using VISA.}
\moduleauthor{Gregor Thalhammer}{gth@users.sourceforge.net}
\moduleauthor{Torsten Bronger}{bronger@physik.rwth-aachen.de}
\sectionauthor{Torsten Bronger}{bronger@physik.rwth-aachen.de}

\section{An example}

Let's go \emph{in medias res} and have a look at a simple example:
\begin{verbatim}
from pyvisa import *

my_instrument = GpibInstrument(14)
my_instrument.write("*IDN?")
print my_instrument.read()
\end{verbatim}
This example already shows the two main design goals of PyVISA: preferring
simplicity over generality, and doing it the object-oriented way.

Every instrument is represented in the source by an object instance.  In this case,
I have a GPIB instrument with instrument number~14, so I create the instance
(i.\,e.\ variable) called \var{my_instrument} accordingly:
\begin{verbatim}
my_instrument = GpibInstrument(14)
\end{verbatim}
Then, I send the message ``*IDN?''\ to the device, which is the standard GPIB
message for ``what are you?''\ or -- in many cases -- ``what's on your display
at the moment?'':
\begin{verbatim}
my_instrument.write("*IDN?")
\end{verbatim}
Finally, I print the instrument's answer on the screen:
\begin{verbatim}
print my_instrument.read()
\end{verbatim}


\section{Module classes}

\subsection{General devices}
\label{sec:general-devices}

\begin{classdesc}{Instrument}{resource_name\optional{, **keyw}}
  represents an instrument, e.\,g.\ a measurement device.  It is independent of
  a particular bus system, i.\,e.\ it may be a GPIB, USB, or whatever
  instrument.  However, it is not possible to perform bus-specific operations
  on instruments created by this class.  For this, have a look at the
  specialised classes like \class{GpibInstrument}
  (section~\ref{sec:gpib-devices}).

  The parameter \var{resource_name} takes the same syntax as resource
  specifiers in VISA\@.  Thus, it begins with the bus system, followed by
  ``\verb|::|'', followed by the location of the device within the bus system,
  and ends with an optional ``\verb|::INSTR|''.

  Possible keyword arguments are:
  \begin{tableii}{ll}{var}{Keyword}{Description}
    \lineii{term_char}{termination characters, see
      section~\ref{sec:termchars}. Default: \code{""} \emph{(empty)}}
    \lineii{timeout}{timeout in seconds for all device operations, see
      section~\ref{sec:timeouts}. Default:~2}
    \lineii{lock}{whether you want to have exclusive access to the device.
      Default: \code{VI_NO_LOCK}}
  \end{tableii}

  \vspace{1ex}
  For further information about the locking mechanism,
  see~\citetitle[http://pyvisa.sourceforge.net/vpp43.html]{The VISA library
    implementation}.
\end{classdesc}

For example, the above mentioned GPIB instrument with number~14 could also be
created with
\begin{verbatim}
my_instrument = Instrument("GPIB::14")
\end{verbatim}
or even more explicitly with
\begin{verbatim}
my_instrument = Instrument("GPIB0::14::INSTR")
\end{verbatim}

The class \class{Instrument} defines the following methods:

\begin{methoddesc}{write}{message}
  writes the string \var{message} to the instrument.
\end{methoddesc}

\begin{methoddesc}{read}{}
  returns a string sent from the instrument.
\end{methoddesc}

The class \class{Instrument} has the following attributes:

\begin{memberdesc}{timeout}
  The timeout in seconds for each I/O operation.  See
  section~\ref{sec:timeouts} for further information.
\end{memberdesc}

\begin{memberdesc}{term_chars}
  The termination characters for each read and write operation.  See
  section~\ref{sec:termchars} for further information.
\end{memberdesc}


\subsection{GPIB devices}
\label{sec:gpib-devices}

\begin{classdesc}{GpibInstrument}{gpib_identifier\optional{,
      board_number\optional{, **keyw}}}
  Since this class is derived from the class \class{Instrument}, please refer
  to section~\ref{sec:general-devices} for the basic operations.  Here I will
  only explain methods and attributes that are special to GPIB devices.
\end{classdesc}

The class \class{GpibInstrument} defines the following methods:

\begin{methoddesc}{wait_for_srq}{\optional{timeout}}
  waits for a serial request (SRQ) coming from the instrument.  Note that this
  method is not ended when \emph{another} instrument signals an SRQ, only
  \emph{this} instrument.
  
  If you don't give a value for \var{timeout} (in seconds), this method waits
  as long as when an SRQ arrives.  Otherwise, \var{timeout} is interpreted as
  the maximal waiting time in seconds.
\end{methoddesc}

\begin{methoddesc}{trigger}{}
  sends a trigger signal to the instrument.
\end{methoddesc}

\bigskip
\begin{classdesc}{Gpib}{\optional{board_number}}
  represents a GPIB board.  Although most setups have at most one GPIB
  interface card or USB-GPIB device (with board number~0), theoretically you
  may have more.  Be that as it may, for board-level operations, i.\,e.\
  operations that affect the whole bus with all connected devices, you must
  create an instance of this class.

  The optional GPIB board number \var{board_number} defaults to~0.
\end{classdesc}

The class \class{Gpib} defines the following methods:

\begin{methoddesc}{send_ifc}{}
  pulses the interface clear line (IFC) for at least 0.1~seconds.
\end{methoddesc}

\begin{raggedright}
You needn't store the board instance in a variable.  Instead, you may send an
IFC signal just by saying \samp{Gpib().send_ifc()}.\par
\end{raggedright}


\subsection{Termination characters}
\label{sec:termchars}

Somehow the computer must detect when the device is finished with sending a
message.  It does so by using different methods, depending on the bus system.
However, you may influence this behaviour by setting termination characters.

Termination characters may be one character or a sequence of characters.
Whenever this character or sequence occurs in the input stream, the read
operation is terminated and the read message is given to the calling
application.  The next read operation continues with the input stream
immediately after the last termination sequence.  In PyVISA, the termination
characters are stripped off the message before it is given to you.

You may set termination characters for each instrument, e.\,g.
\begin{verbatim}
my_instrument.term_chars = "\r"
\end{verbatim}
Alternatively you can give it when creating your instrument object:
\begin{verbatim}
my_instrument = Instrument("GPIB::10", term_chars = "\r")
\end{verbatim}
The default value depends on the bus system.  Generally, the sequence is empty.
For GPIB instruments however, it's set to \verb|"\r\n"| (i.\,e.~``CR\,LF'')\@.
For RS232 (once it's supported), it's just \verb|"\n"|~(LF)\@.

You may add two options to the termination characters: \code{NOEND} and
\code{DELAY}.  They are inspired by HT~Basic.  \code{NOEND} makes only sense
for GPIB instruments.  It switches off detection of the end of a message using
the EOI bus line.  (This means that by default the observance of this line is
on.)

\code{DELAY} takes a numeric argument that denotes the time in seconds to wait
after each write operation.  So you could write:
\begin{verbatim}
my_instrument = Instrument("GPIB::10", term_chars = "\r NOEND DELAY 1.2")
\end{verbatim}
This will set the delay to 1.2~seconds, and the EOI line is ignored.  Note that
spaces before the options are ignored, and that at least one space must be
there, unless the termination sequence itself is empty.  So, \code{"\e
  rNOEND"}, \code{" NOEND"}, and \code{"noend"} are all illegal in the sense
that they will be interpreted as one long termination sequence, which is
probably not what you want.  The correct variants are \code{"\e r NOEND"},
\code{"NOEND"}, and \code{"NOEND"}.


\subsection{Timeouts}
\label{sec:timeouts}

Very most VISA I/O operations may be performed with a timeout.  If a timeout is
set, every operation that takes longer than the timeout is aborted and an
exception is raised.  Timeouts are given per instrument.

For all PyVISA objects, a timeout is set with
\begin{verbatim}
my_device.timeout = 25
\end{verbatim}
Here, \var{my_device} may be a device, an interface or whatever, and its
timeout is set to 25~seconds.  Floating-point values are allowed.  If you set
it to zero, all operations must succeed instantaneously.  You remove a timeout
with
\begin{verbatim}
del my_device.timeout
\end{verbatim}
Now every operation of the resource takes as long as it takes, even
indefinitely if necessary.

The default timeout is 2~seconds, but you can change it when creating the
device object:
\begin{verbatim}
my_instrument = Instrument("ASRL1", timeout = 5)
\end{verbatim}
This creates the object variable \var{my_instrument} and sets its timeout to
5~seconds.  In this and only in this context, a timeout value of \code{None} is
allowed which removes the timeout for this device.

\section{Module functions}

\begin{funcdesc}{get_instruments_list}{\optional{use_aliases}}
  returns a list with all instruments that are known to the local VISA system.
  If you're lucky, these are all instruments connected with the computer.

  The boolean \var{use_aliases} is \code{True} by default, which means that the
  more human-friendly aliases like ``\code{COM1}'' instead of ``\code{ASRL1}''
  are returned.  With some VISA systems\footnote{such as the ``Measurement and
    Automation Center'' by National Instruments} you can define your own
  aliases for each device, e.\,g.\ ``\code{keithley617}'' for
  ``\code{GPIB0::15::INSTR}''.  If \var{use_aliases} is \code{False}, only
  standard resource names are returned.
\end{funcdesc}

\section{Exceptions}


\section{Mixing with direct VISA commands}

You can mix the high-level object-oriented approach described in this document
with middle-level VISA function calls in module \module{vpp43} as described in
\citetitle[http://pyvisa.sourceforge.net/vpp43.html]{The VISA library
  implementation} which is also part of the PyVISA package.  By doing so you
have full control of your VISA implementation.

The VISA functions need to know what session you are referring to.  PyVISA
opens exactly one session for each instrument or interface and stores its
session handle in the instance attribute~\member{vi}.  Thus you can write
\begin{verbatim}
from pyvisa import *
my_instrument = GpibInstrument(14)
vpp43.clear(my_instrument.vi)
\end{verbatim}

In case you need the session handle for the default resource manager, it's
stored in \member{resource_manager.session}:
\begin{verbatim}
from pyvisa import *
my_instrument_handle = vpp43.open(resource_manager.session, "GPIB::14",
                                  VI_EXCLUSIVE_LOCK)
\end{verbatim}

\section{Installation}

\subsection{Prerequisites}

The PyVISA package doesn't include a low-level VISA implementation itself.  You
have to download it from one of the VISA vendors, e.\,g.\ from the
\ulink{National Instruments VISA pages}{http://ni.com/visa/}.  Please install
it properly before you proceed.

Additionally, your Python installation needs a fresh version of
\ulink{ctypes}{http://starship.python.net/crew/theller/ctypes/}.  By the way,
if you use Windows, it may be convenient to install \ulink{Enthought
  Python}{http://www.enthought.com/python/}.  It is a special Python version
with all-included philosophy for scientific and engineering
applications.\footnote{Of course, it's highly advisable not to have installed
  another version of Python on your system before you install Enthought
  Python.}


\subsection{Setting up the module}

\subsubsection{Windows}

PyVISA expects a file called \file{visa32.dll} in the \envvar{PATH}\@.  You
must assure that this is the case.  For example, on my system you find this
file in \file{C:\e WINNT\e system32\e}.  Either copy it there or expand your
\envvar{PATH}\@.


\subsubsection{Linux}

For Linux, the default VISA library is
\file{/usr/local/vxipnp/linux/bin/libvisa.so.7}.  However, here you have the
choice.  If it's called differently on your system, or is placed in another
directory, you can tell PyVISA so by saying
\begin{verbatim}
import vpp43
vpp43.visa_library.load_library("/usr/local/visa/libvisa.so.7")
from pyvisa import *
...
\end{verbatim}
Of course, you must set the path according to your installation.


\section{About PyVISA}

PyVISA was originally programmed by Gregor Thalhammer, Innsbruck/Austria, and
Torsten Bronger, Aachen/\hskip0pt Germany.  It bases on earlier experiences by
Thalhammer.

Its homepage is \url{http://sourceforge.net/projects/pyvisa/}.  Please report
bugs there.

\end{document}

% LocalWords:  ascii british PyVISA pyvisa keyw GpibInstrument var NOEND EOI
% LocalWords:  rNOEND noend ASRL keithley gpib Center vpp ctypes dll WINNT IDN
% LocalWords:  ifc
