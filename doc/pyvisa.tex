% -*- mode: LaTeX; coding: utf-8; ispell-local-dictionary: "british"; -*-
% 
%   pyvisa.tex - Manual for high-level OO instrument access on top of VISA
% 
%   Copyright © 2005 Gregor Thalhammer <gth@users.sourceforge.net>,
%                    Torsten Bronger <bronger@physik.rwth-aachen.de>.
% 
%   This file is part of PyVISA.
% 
%   PyVISA is free software; you can redistribute it and/or modify it under
%   the terms of the GNU General Public License as published by the Free
%   Software Foundation; either version 2 of the License, or (at your option)
%   any later version.
% 
%   PyVISA is distributed in the hope that it will be useful, but WITHOUT ANY
%   WARRANTY; without even the implied warranty of MERCHANTABILITY or FITNESS
%   FOR A PARTICULAR PURPOSE.  See the GNU General Public License for more
%   details.
% 
%   You should have received a copy of the GNU General Public License along
%   with PyVISA; if not, write to the Free Software Foundation, Inc., 59
%   Temple Place, Suite 330, Boston, MA 02111-1307 USA
% 

\documentclass{howto}

\title{PyVISA}
\release{0.9.4}

\author{Torsten Bronger}
\authoraddress{
	Aachen, Germany\\
	\email{bronger@physik.rwth-aachen.de}
}

\date{22 Juli 2005}

\makeindex

\begin{document}

\maketitle

\ifhtml
\chapter*{Front Matter\label{front}}
\fi

Copyright \copyright{} 2005 Torsten Bronger.\\
Permission is granted to copy, distribute and/or modify this document under the
terms of the GNU Free Documentation License, Version~1.2 or any later version
published by the Free Software Foundation; with no Invariant Sections, no
Front-Cover Texts, and no Back-Cover Texts.  A copy of the license is included
as a separate file \file{LICENSE} in the PyVISA distribution.

\begin{abstract}

\noindent
PyVISA enables you to control your measurement and test equipment -- digital
multimeters, motors, sensors and the like.  This document covers the
easy-to-use \module{visa} module of the PyVISA package.  It implements control
of measurement devices in a straightforward and convenient way.  The design
goal is to combine HTBasic's simplicity with Python's modern syntax and
powerful set of libraries.

PyVISA doesn't implement VISA itself.  Instead, PyVISA provides bindings to the
VISA library (a DLL or ``shared object'' file).  This library is usually
shipped with your GPIB interface.  If not, you can download it from your
interface equipment vendor (National Instruments, Agilent, etc).

PyVISA is free software under the terms of the GPL\@.  It can be downloaded at
\ulink{PyVISA's project page}{http://sourceforge.net/projects/pyvisa}.  You can
report bugs there, too.  This package is still not thoroughly tested.
Therefore, I need feedback from people who've given it a try.  It works for us,
but we own only a limited set of measurement equipment.
\end{abstract}

\begin{samepage}
  \tableofcontents
\end{samepage}

\declaremodule{extension}{pyvisa}
  \platform{Linux,Windows}
\modulesynopsis{Controlling measurement and test equipment using VISA.}
\moduleauthor{Gregor Thalhammer}{gth@users.sourceforge.net}
\moduleauthor{Torsten Bronger}{bronger@physik.rwth-aachen.de}
\sectionauthor{Torsten Bronger}{bronger@physik.rwth-aachen.de}

\section{An example}

Let's go \emph{in medias res} and have a look at a simple example:
\begin{verbatim}
from visa import *

my_instrument = GpibInstrument(14)
my_instrument.write("*IDN?")
print my_instrument.read()
\end{verbatim}
This example already shows the two main design goals of PyVISA: preferring
simplicity over generality, and doing it the object-oriented way.

Every instrument is represented in the source by an object instance.  In this case,
I have a GPIB instrument with instrument number~14, so I create the instance
(i.\,e.\ variable) called \var{my_instrument} accordingly:
\begin{verbatim}
my_instrument = GpibInstrument(14)
\end{verbatim}
Then, I send the message ``*IDN?''\ to the device, which is the standard GPIB
message for ``what are you?''\ or -- in some cases -- ``what's on your display
at the moment?'':
\begin{verbatim}
my_instrument.write("*IDN?")
\end{verbatim}
Finally, I print the instrument's answer on the screen:
\begin{verbatim}
print my_instrument.read()
\end{verbatim}


\subsection{Example for serial (RS232) device}

The only RS\,232 device in my lab is an old Oxford ITC\,4 temperature
controller, which is connected through COM\,2 with my computer.  The following
code prints its self-identification on the screen:
\begin{verbatim}
from visa import *

itc4 = Instrument("COM2", term_chars = CR)
itc4.write("V")
print itc4.read()
\end{verbatim}
Instead of separate write and read operations, you can do both with one
\code{ask()} call.  Thus, the above source code is equivalent to
\begin{verbatim}
from visa import *

itc4 = Instrument("COM2", term_chars = CR)
print itc4.ask("V")
\end{verbatim}
It couldn't be simpler.


\subsection{A more complex example}
\label{sec:more-complex-example}

The following example shows how to use SCPI commands with a Keithley 2000
multimeter in order to measure 10~voltages.  After having read them, the
program calculates the average voltage and prints it on the screen.

I'll explain the program step-by-step.  First, we have to initialise the
instrument:
\begin{verbatim}
from visa import GpibInstrument

keithley = GpibInstrument(12)
keithley.write("*rst; status:preset; *cls")
\end{verbatim}
Here, we create the instrument variable \var{keithley}, which is used for all
further operations on the instrument.  Immediately after it, we send the
initialisation and reset message to the instrument.

The next step is to write all the measurement parameters, in particular the
interval time (500\,ms) and the number of readings~(10) to the instrument.  I
won't explain it in detail.  Have a look at an SCPI and/or Keithley~2000
manual.
\begin{verbatim}
interval_in_ms = 500
number_of_readings = 10

keithley.write("status:measurement:enable 512; *sre 1")
keithley.write("sample:count %d" % number_of_readings)
keithley.write("trigger:source bus")
keithley.write("trigger:delay %f" % (interval_in_ms / 1000.0))

keithley.write("trace:points %d" % number_of_readings)
keithley.write("trace:feed sense1; feed:control next")
\end{verbatim}
Okay, now the instrument is prepared to do the measurement.  The next three
lines make the instrument waiting for a trigger pulse, trigger it, and wait
until it sends a ``service request'':
\begin{verbatim}
keithley.write("initiate")
keithley.trigger()
keithley.wait_for_srq()
\end{verbatim}
With sending the service request, the instrument tells us that the measurement
has been finished and that the results are ready for transmission.  We could
read them with \samp{keithley.ask("trace:data?")}\ however, then we'd get
\begin{verbatim}
NDCV-000.0004E+0,NDCV-000.0005E+0,NDCV-000.0004E+0,NDCV-000.0007E+0,
NDCV-000.0000E+0,NDCV-000.0007E+0,NDCV-000.0008E+0,NDCV-000.0004E+0,
NDCV-000.0002E+0,NDCV-000.0005E+0
\end{verbatim}
which we would have to convert to a Python list of numbers.  Fortunately, the
\code{ask_for_values()} method does this work for us:
\begin{verbatim}
voltages = keithley.ask_for_values("trace:data?")
print "Average voltage: ", sum(voltages) / len(voltages)
\end{verbatim}
Finally, we should reset the instrument's data buffer and SRQ status register,
so that it's ready for a new run.  Again, this is explained in detail in the
instrument's manual:
\begin{verbatim}
keithley.ask("status:measurement?")
keithley.write("trace:clear; feed:control next")
\end{verbatim}
That's it.  18 lines of lucid code.  (Well, SCPI is awkward, but that's another
story.)


\section{Module classes}

\subsection{General devices}
\label{sec:general-devices}

\begin{classdesc}{Instrument}{resource_name\optional{, **keyw}}
  represents an instrument, e.\,g.\ a measurement device.  It is independent of
  a particular bus system, i.\,e.\ it may be a GPIB, serial, USB, or whatever
  instrument.  However, it is not possible to perform bus-specific operations
  on instruments created by this class.  For this, have a look at the
  specialised classes like \class{GpibInstrument}
  (section~\ref{sec:gpib-devices}).

  The parameter \var{resource_name} takes the same syntax as resource
  specifiers in VISA\@.  Thus, it begins with the bus system followed by
  ``\verb|::|'', continues with the location of the device within the bus
  system, and ends with an optional ``\verb|::INSTR|''.

  Possible keyword arguments are:
  \begin{tableii}{ll}{var}{Keyword}{Description}
    \lineii{timeout}{timeout in seconds for all device operations, see
      section~\ref{sec:timeouts}. Default:~5}
    \lineii{chunk_size}{Length of read data chunks in bytes, see
      section~\ref{sec:chunk-length}. Default: 20\,kB}
    \lineii{values_format}{Data format for lists of read values, see
      section~\ref{sec:reading-binary-data}. Default: \code{ascii}}
    \lineii{term_char}{termination characters, see
      section~\ref{sec:termchars}. Default: \code{""} \emph{(empty)}}
    \lineii{send_end}{whether to assert END after each write operation, see
      section~\ref{sec:termchars}. Default: \code{True}}
    \lineii{delay}{delay in seconds after each write operation, see
      section~\ref{sec:termchars}. Default:~0}
    \lineii{lock}{whether you want to have exclusive access to the device.
      Default: \code{VI_NO_LOCK}}
  \end{tableii}

  \vspace{1ex}
  For further information about the locking mechanism,
  see~\citetitle[http://pyvisa.sourceforge.net/vpp43.html]{The VISA library
    implementation}.
\end{classdesc}

For example, the above mentioned GPIB instrument with number~14 could also be
created with
\begin{verbatim}
my_instrument = Instrument("GPIB::14")
\end{verbatim}
or even more explicitly with
\begin{verbatim}
my_instrument = Instrument("GPIB0::14::INSTR")
\end{verbatim}

The class \class{Instrument} defines the following methods and attributes:

\begin{methoddesc}{write}{message}
  writes the string \var{message} to the instrument.
\end{methoddesc}

\begin{methoddesc}{read}{}
  returns a string sent from the instrument to the computer.
\end{methoddesc}

\begin{methoddesc}{read_floats}{\optional{format}}
  returns a list of decimal values (floats) sent from the instrument to the
  computer.  See section~\ref{sec:more-complex-example} above.  The list may
  contain only one element or may be empty.

  The optional \var{format} argument overrides the setting of
  \var{values_format}.  For information about that, see
  section~\ref{sec:reading-binary-data}.
\end{methoddesc}

\begin{methoddesc}{ask}{message}
  sends the string \var{message} to the instrument and returns the answer
  string from the instrument.
\end{methoddesc}

\begin{methoddesc}{ask_for_floats}{message\optional{, format}}
  sends the string \var{message} to the instrument and reads the answer as a
  list of values, just as \code{read_floats()} does.

  The optional \var{format} argument overrides the setting of
  \var{values_format}.  For information about that, see
  section~\ref{sec:reading-binary-data}.
\end{methoddesc}

\begin{methoddesc}{clear}{}
  resets the device.  This operation is highly bus-dependent.  I refer you to
  the original VISA documentation, which explains how this is achieved for VXI,
  GPIB, serial, etc.
\end{methoddesc}

\begin{methoddesc}{trigger}{}
  sends a trigger signal to the instrument.
\end{methoddesc}

\begin{methoddesc}{read_raw}{}
  returns a string sent from the instrument to the computer.  In contrast to
  \code{read()}, no termination characters are checked or stripped.  You get
  the pristine message.
\end{methoddesc}

\begin{memberdesc}{timeout}
  The timeout in seconds for each I/O operation.  See
  section~\ref{sec:timeouts} for further information.
\end{memberdesc}

\begin{memberdesc}{term_chars}
  The termination characters for each read and write operation.  See
  section~\ref{sec:termchars} for further information.
\end{memberdesc}

\begin{memberdesc}{send_end}
  Whether or not to assert EOI (or something equivalent, depending on the
  interface type) after each write operation.  See section~\ref{sec:termchars}
  for further information.
\end{memberdesc}

\begin{memberdesc}{delay}
  Time in seconds to wait after each write operation.  See
  section~\ref{sec:termchars} for further information.
\end{memberdesc}

\begin{memberdesc}{values_format}
  The format for multi-value data sent from the instrument to the computer.
  See section~\ref{sec:reading-binary-data} for further information.
\end{memberdesc}


\subsection{GPIB devices}
\label{sec:gpib-devices}

\begin{classdesc}{GpibInstrument}{gpib_identifier\optional{,
      board_number\optional{, **keyw}}}
  represents a GPIB instrument.  If \var{gpib_identifier} is a string, it is
  interpreted as a VISA resource name.  If it is a number, it denotes the
  device number at the GPIB bus.

  The optional \var{board_number} defaults to zero.  If you have more that one
  GPIB bus system attached to the computer, you can select the bus with this
  parameter.

  The keyword arguments are interpreted the same as with the class
  \class{Instrument}.
\end{classdesc}

\begin{notice}
  Since this class is derived from the class \class{Instrument}, please refer
  to section~\ref{sec:general-devices} for the basic operations.
  \class{GpibInstrument} can do everything that \class{Instrument} can do, so
  it simply extends the original class with GPIB-specific operations.
\end{notice}

The class \class{GpibInstrument} defines the following methods:

\begin{methoddesc}{wait_for_srq}{\optional{timeout}}
  waits for a serial request (SRQ) coming from the instrument.  Note that this
  method is not ended when \emph{another} instrument signals an SRQ, only
  \emph{this} instrument.
  
  The \var{timeout} argument, given in seconds, denotes the maximal waiting
  time.  The default value is 25~(seconds).  If you pass \code{None} for the
  timeout, this method waits forever if no SRQ arrives.
\end{methoddesc}

\bigskip
\begin{classdesc}{Gpib}{\optional{board_number}}
  represents a GPIB board.  Although most setups have at most one GPIB
  interface card or USB-GPIB device (with board number~0), theoretically you
  may have more.  Be that as it may, for board-level operations, i.\,e.\
  operations that affect the whole bus with all connected devices, you must
  create an instance of this class.

  The optional GPIB board number \var{board_number} defaults to~0.
\end{classdesc}

The class \class{Gpib} defines the following method:

\begin{methoddesc}{send_ifc}{}
  pulses the interface clear line (IFC) for at least 0.1~seconds.
\end{methoddesc}

\begin{notice}
You needn't store the board instance in a variable.  Instead, you may send an
IFC signal just by saying \samp{Gpib().send_ifc()}.
\end{notice}


\subsection{Timeouts}
\label{sec:timeouts}

Very most VISA I/O operations may be performed with a timeout.  If a timeout is
set, every operation that takes longer than the timeout is aborted and an
exception is raised.  Timeouts are given per instrument.

For all PyVISA objects, a timeout is set with
\begin{verbatim}
my_device.timeout = 25
\end{verbatim}
Here, \var{my_device} may be a device, an interface or whatever, and its
timeout is set to 25~seconds.  Floating-point values are allowed.  If you set
it to zero, all operations must succeed instantaneously.  You must not set it
to \code{None}.  Instead, if you want to remove the timeout, just say
\begin{verbatim}
del my_device.timeout
\end{verbatim}
Now every operation of the resource takes as long as it takes, even
indefinitely if necessary.

The default timeout is 5~seconds, but you can change it when creating the
device object:
\begin{verbatim}
my_instrument = Instrument("ASRL1", timeout = 8)
\end{verbatim}
This creates the object variable \var{my_instrument} and sets its timeout to
5~seconds.  In this context, a timeout value of \code{None} is allowed, which
removes the timeout for this device.


\subsection{Chunk length}
\label{sec:chunk-length}

If you read data from a device, you must store it somewhere.  Unfortunately,
PyVISA must make space for the data \emph{before} it starts reading, which
means that it must know how much data the device will send.  However, it
doesn't know a~priori.

Therefore, PyVISA reads from the device in \emph{chunks}.  Each chunk is
20~kilobytes long by default.  If there's still data to be read, PyVISA repeats
the procedure and eventually concatenates the results and returns it to you.
Those 20~kilobytes are large enough so that mostly one read cycle is
sufficient.

The whole thing happens automatically, as you can see.  Normally you needn't
worry about it.  However, some devices don't like to send data in chunks.  So
if you have trouble with a certain device and expect data lengths larger than
the default chunk length, you should increase its value by saying e.\,g.
\begin{verbatim}
my_instrument.chunk_size = 102400
\end{verbatim}
This example sets it to 100~kilobytes.


\subsection{Reading binary data}
\label{sec:reading-binary-data}

Some instruments allow for sending the measured data in binary form.  This has
the advantage that the data transfer is much shorter and takes less time.
PyVISA currently supports three forms of transfers:

\begin{description}
\item[ascii] This is the default mode.  It assumes a normal string with comma-
  or whitespace-separated values.
\item[single] The values are expected as a binary sequence of IEEE floating
  point values with single precision (i.\,e.\ four bytes each).\footnote{All
    flavours of binary data streams defined in IEEE\,488.2 are supported,
    i.\,e.\ those beginning with
    ``\textit{$\langle$header$\rangle$}~\texttt{\#}\textit{$\langle$digit$\rangle$}'',
    where \textit{$\langle$header$\rangle$} is optional, and
    \textit{$\langle$digit$\rangle$} may also be~``0''.}
\item[double] The same as \textbf{single}, but with values of double precision
  (eight bytes each).
\end{description}

You can set the form of transfer with the property \code{values_format}, either
with the generation of the object,
\begin{verbatim}
my_instrument = GpibInstrument(12, values_format = single)
\end{verbatim}
or later by setting the property directly:
\begin{verbatim}
my_instrument.values_format = single
\end{verbatim}
Setting this option affects the methods \code{read_values()} and
\code{ask_for_values()}.

In some cases it may be necessary to set the \emph{byte order}, also known as
\emph{endianness}.  PyVISA assumes little-endian as default.  Some instruments
call this ``swapped'' byte order.  However, there is also big-endian byte
order.  In this case you have to append \samp{| big_endian} to your values
format:
\begin{verbatim}
my_instrument = GpibInstrument(12, values_format = single | big_endian)
\end{verbatim}


\subsubsection{Example}
\label{sec:binary-example}

In order to demonstrate how easy reading binary data can be, remember our
example from section~\ref{sec:more-complex-example}.  You just have to append
the lines
\begin{verbatim}
keithley.write("format:data sreal")
keithley.values_format = single
\end{verbatim}
to the initialisation commands, and all measurement data will be transmitted as
binary.  You will only notice the increased speed, as PyVISA converts it into
the same list of values as before.


\subsection{Termination characters}
\label{sec:termchars}

Somehow the computer must detect when the device is finished with sending a
message.  It does so by using different methods, depending on the bus system.
In most cases you don't need to worry about termination characters.  In our
institute, only the RS\,232 devices tend to be somewhat fussy about them.
However, if you have trouble, you may influence termination characters with
PyVISA\@.

Termination characters may be one character or a sequence of characters.
Whenever this character or sequence occurs in the input stream, the read
operation is terminated and the read message is given to the calling
application.  The next read operation continues with the input stream
immediately after the last termination sequence.  In PyVISA, the termination
characters are stripped off the message before it is given to you.

You may set termination characters for each instrument, e.\,g.
\begin{verbatim}
my_instrument.term_chars = CR
\end{verbatim}
Alternatively you can give it when creating your instrument object:
\begin{verbatim}
my_instrument = Instrument("GPIB::10", term_chars = CR)
\end{verbatim}
The default value depends on the bus system.  Generally, the sequence is empty,
in particular for GPIB\@.  For RS232 (once it's got its own class), it'll
be~\code{LF}\@.

The termination characters sequence is an ordinary string.  \code{CR} and
\code{LF} are just string constants that allow readable access to \verb|"\r"|
and~\verb|"\n"|.  Therefore, instead of \code{CR+LF}, you can also write
\verb|"\r\n"|, whichever you like more.

There are two further options related to this topic, namely \code{send_end} and
\code{delay}.  \code{send_end} is a boolean.  If it's \code{True} (the
default), the EOI line is asserted after each write operation, signalling the
end of the operation.  EOI is GPIB-specific but similar action is taken for
other interfaces.

The argument \code{delay} is the time in seconds to wait after each write
operation.  So you could write:
\begin{verbatim}
my_instrument = Instrument("GPIB::10", term_chars = CR, send_end = False,
                           delay = 1.2")
\end{verbatim}
This will set the delay to 1.2~seconds, and the EOI line is omitted.  By the
way, omitting EOI is \emph{not} recommended, so if you omit it nevertheless,
you should know what you're doing.


\section{Module functions}

\begin{funcdesc}{get_instruments_list}{\optional{use_aliases}}
  returns a list with all instruments that are known to the local VISA system.
  If you're lucky, these are all instruments connected with the computer.

  The boolean \var{use_aliases} is \code{True} by default, which means that the
  more human-friendly aliases like ``\code{COM1}'' instead of ``\code{ASRL1}''
  are returned.  With some VISA systems\footnote{such as the ``Measurement and
    Automation Center'' by National Instruments} you can define your own
  aliases for each device, e.\,g.\ ``\code{keithley617}'' for
  ``\code{GPIB0::15::INSTR}''.  If \var{use_aliases} is \code{False}, only
  standard resource names are returned.
\end{funcdesc}


\section{Mixing with direct VISA commands}

You can mix the high-level object-oriented approach described in this document
with middle-level VISA function calls in module \module{vpp43} as described in
\citetitle[http://pyvisa.sourceforge.net/vpp43.html]{The VISA library
  implementation} which is also part of the PyVISA package.  By doing so, you
have full control of your devices.  I recommend to import the VISA functions
with
\begin{verbatim}
from pyvisa import vpp43
\end{verbatim}
Then you can use them with \samp{vpp43.\emph{function_name}(...)}.

The VISA functions need to know what session you are referring to.  PyVISA
opens exactly one session for each instrument or interface and stores its
session handle in the instance attribute~\member{vi}.  For example, these two
lines are equivalent:
\begin{verbatim}
my_instrument.clear()
vpp43.clear(my_instrument.vi)
\end{verbatim}

In case you need the session handle for the default resource manager, it's
stored in \member{resource_manager.session}:
\begin{verbatim}
from visa import *
from pyvisa import vpp43
my_instrument_handle = vpp43.open(resource_manager.session, "GPIB::14",
                                  VI_EXCLUSIVE_LOCK)
\end{verbatim}

\section{Installation}

\subsection{Prerequisites}

PyVISA needs Python version~2.3 or newer.

The PyVISA package doesn't include a low-level VISA implementation itself.  You
have to download it from one of the VISA vendors, e.\,g.\ from the
\ulink{National Instruments VISA pages}{http://ni.com/visa/}.  Please install
it properly before you proceed.

Additionally, your Python installation needs a fresh version of
\ulink{ctypes}{http://starship.python.net/crew/theller/ctypes/}.  By the way,
if you use Windows, it may be convenient to install \ulink{Enthought
  Python}{http://www.enthought.com/python/}.  It is a special Python version
with all-included philosophy for scientific and engineering
applications.\footnote{Of course, it's highly advisable not to have installed
  another version of Python on your system before you install Enthought
  Python.}


\subsection{Setting up the module}

\subsubsection{Windows}

PyVISA expects a file called \file{visa32.dll} in the \envvar{PATH}\@.  For
example, on my system you find this file in \file{C:\e WINNT\e system32\e}.
Either copy it there or expand your \envvar{PATH}.  Alternatively, you can
create an INI file.  You must do this anyway if the file is not called
\file{visa32.dll} on your system.


\subsubsection{Linux}

For Linux, the VISA library is by default at
\file{/usr/local/vxipnp/linux/bin/libvisa.so.7}.  If this is not the case on
your installation, you have to create an INI file.


\subsubsection{INI file for customisation}

If the VISA library file is not at the default place, or doesn't have the
default name for your operating system (see above), you can tell PyVISA by
creating a file called \file{.pyvisarc} (mind the leading dot).

Another motivation for setting up an INI file is that you have more than one
VISA library, e.\,g.\ because two GPIB interfaces of two different vendors are
connected with the computer.  However, in this case I'd try to use both
interfaces with one library because sometimes you're lucky and it works.  Note
that PyVISA is currently not able to switch between DLLs while the program is
running.

For Windows, place it in your ``Documents and Settings'' folder,\footnote{its
  name depends on the language of your Windows version} e.\,g.\
\begin{verbatim}
C:\Documents and Settings\smith\.pyvisarc
\end{verbatim}
if ``smith'' is the name of your login account.  For Linux, put it in your home
directory.

This file has the format of an INI~file.  For example, if the library is at
\file{/usr/lib/libvisa.so.7}, the file \file{.pyvisarc} must contain the
following:
\begin{verbatim}
[Paths]

VISA library: /usr/lib/libvisa.so.7
\end{verbatim}
Please note that ``\code{[Paths]}'' is treated case-sensitively.

You can define a site-wide configuration file at
\file{/usr/share/pyvisa/.pyvisarc}.  (It may also be \file{/usr/local/\ldots}\
depending on the location of your Python.)  Under Windows, this file is usually
placed at \file{c:\e Python23\e share\e pyvisa\e .pyvisarc}.


\section{About PyVISA}

PyVISA was originally programmed by Gregor Thalhammer, Innsbruck/Austria, and
Torsten Bronger, Aachen/\hskip0pt Germany.  It bases on earlier experiences by
Thalhammer.

Its homepage is \url{http://sourceforge.net/projects/pyvisa/}.  Please report
bugs there.  \textbf{I'm also very keen to know whether PyVISA works for you or
  not.  Thank you!}  

\end{document}

% LocalWords:  ascii british PyVISA pyvisa keyw GpibInstrument var NOEND EOI
% LocalWords:  rNOEND noend ASRL keithley gpib Center vpp ctypes dll WINNT IDN
% LocalWords:  ifc SCPI srq utf tex pyvisarc INI ITC DLLs
