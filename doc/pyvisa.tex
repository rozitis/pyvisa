% -*- mode: LaTeX; coding: us-ascii; ispell-local-dictionary: "british"; -*-
\documentclass{howto}

\title{PyVISA\\[0.5ex]\large measurement and test equipment control with Python}
\release{0.9}

\author{Torsten Bronger}
\authoraddress{
	Aachen, Germany\\
	Email: \email{bronger@physik.rwth-aachen.de}
}

\date{17 June 2005}

\makeindex

\begin{document}

\maketitle

\ifhtml
\chapter*{Front Matter\label{front}}
\fi

Copyright \copyright{} 2005 Torsten Bronger.  All rights reserved.

\begin{abstract}

\noindent
This document covers the PyVISA module.  It bases on the VISA library and
implements control of measurement devices in a convenient way.
\end{abstract}

\tableofcontents

\declaremodule{extension}{pyvisa}
  \platform{Linux,Windows}
\modulesynopsis{Controlling measurement and test equipment using VISA.}
\moduleauthor{Gregor Thalhammer}{gth@users.sourceforge.net}
\moduleauthor{Torsten Bronger}{bronger@physik.rwth-aachen.de}
\sectionauthor{Torsten Bronger}{bronger@physik.rwth-aachen.de}

\section{An example}

Let's go \emph{in medias res} and have a look at a simple example:
\begin{verbatim}
from pyvisa import *

my_instrument = GpibInstrument(14)
my_instrument.write("*IDN?")
print my_instrument.read()
\end{verbatim}
This example already shows the two main design goals of PyVISA: preferring
simplicity over generality, and doing it the object-oriented way.

Every instrument is represented in the source by an object instance.  In this case,
I have a GPIB instrument with instrument number~14.

\section{Module classes}

\begin{classdesc}{Instrument}{resource_name\optional{, **keyw}}
  represents an instrument, e.\,g.\ a measurement device.  It is independent of
  a particular bus system, i.\,e.\ it may be a GPIB, USB, or whatever
  instrument.  However, it is not possible to perform bus-specific operations
  on instruments created by this class.  For this, have a look at the
  specialised classes like \class{GpibInstrument}.

  The \var{resource_name} takes the same syntax as resource specifiers in
  VISA\@.  Thus, it begins with the bus system, followed by ``\verb|::|'',
  followed by the location of the device within the bus system, and ended with
  an optional ``\verb|::INSTR|''.

  Possible keyword arguments are:
  \begin{tableii}{ll}{var}{Keyword}{Description}
    \lineii{term_char}{termination characters, see
      section~\ref{sec:termchars}. Default: \code{""} \emph{(empty)}}
    \lineii{timeout}{timeout in seconds for all device operations, see
      section~\ref{sec:timeouts}. Default:~2}
  \end{tableii}
\end{classdesc}

For example, the above mentioned GPIB instrument with number~14 could also be
created with
\begin{verbatim}
my_instrument = Instrument("GPIB::14")
\end{verbatim}
or even more explicitly with
\begin{verbatim}
my_instrument = Instrument("GPIB0::14::INSTR")
\end{verbatim}

The class \class{Instrument} defines the following methods:

\begin{methoddesc}{write}{message}
  writes the string \var{message} to the instrument.
\end{methoddesc}

\begin{methoddesc}{read}{}
  returns a string send from the instrument.
\end{methoddesc}


\subsection{Termination characters}
\label{sec:termchars}

Somehow the computer must detect when the device is finished with sending a
message.  It does so by using different methods, depending on the bus system.
However, you may influence this behaviour by setting termination characters.

Termination characters may be one character or a sequence of characters.
Whenever this character or sequence occurs in the input stream, the read
operation is terminated and the read message is given to the calling
application.  The next read operation continues with the imput stream
immediately after the last termination characters.

In PyVISA, the termination characters are stripped off the message before it is
given to you.

You may set termination characters for each instrument, e.\,g.
\begin{verbatim}
my_instrument.term_chars = "\r"
\end{verbatim}
Alternatively you can give it when creating your instrument object:
\begin{verbatim}
my_instrument = Instrument("GPIB::10", term_chars = "\r")
\end{verbatim}
The default value depends on the bus system.  Generally, the sequence is empty.
For GPIB instruments however, it's set to \verb|"\r\n"| (i.\,e.~``CR\,LF'')\@.
For RS232 (once it's supported), it's just \verb|"\n"|~(LF)\@.

You may add two options to the termination characters: \code{NOEND} and
\code{DELAY}.  They are inspired from

\subsection{Timeouts}
\label{sec:timeouts}

Very most VISA I/O operations may be performed with a timeout.  If a timeout is
set, every operation that takes longer than the timeout is aborted and an
exception is raised.  Timeouts are given per instrument.

For all PyVISA objects, a timeout is set with
\begin{verbatim}
my_resource.timeout = 25
\end{verbatim}
Here, \var{my_resource} may be a device, an interface or whatever, and its
timeout is set to 25~seconds.  Floating-point values are allowed.  If you set
it to zero, all operations must succeed instantaneously.  You remove a timeout
with
\begin{verbatim}
del my_resource.timeout
\end{verbatim}
Now every operation of the resource is tried as long as it takes, even
indefinitely if necessary.

The default timeout is 2~seconds, but you can change it when creating the
device object:
\begin{verbatim}
my_instrument = Instrument("ASRL1", timeout = 5)
\end{verbatim}
This creates \var{my_instrument} and sets its timeout to 5~seconds.  In this
and only in this context, a timeout value of \code{None} is allowed which
removes the timeout for this device.

\section{Module functions}

\section{Exceptions}


\section{Installation}


\section{About PyVISA}

PyVISA was originally programmed by Gregor Thalhammer, Innsbruck/Austria, and
Torsten Bronger, Aachen/\hskip0pt Germany.  It bases on earlier experiences by
Thalhammer.

Its homepage is \url{http://sourceforge.net/projects/pyvisa/}.  Please report
bugs there.

\end{document}

% LocalWords:  ascii british PyVISA pyvisa keyw GpibInstrument var
